\chapter{Objetivos}

El objetivo principal de este Trabajo de Fin de Grado es el diseño y desarrollo de un sistema capaz de registrar y analizar la interacción del usuario en entornos de modelado tridimensional, utilizando Blender como plataforma de referencia.

\section{Objetivos específicos}

Para alcanzar este objetivo general, se han definido los siguientes objetivos específicos:

\begin{itemize}
    \item MDiseñar un sistema de recopilación de datos no intrusivo que funcione de manera transparente para el usuario durante el proceso de modelado 3D.
    \item Implementar un addon para Blender capaz de registrar información temporal, espacial y estructural de la escena y de los objetos modelados.
    \item Detectar de forma automática eventos relevantes de modelado, tales como duplicados, fusiones de geometría, cambios de modo de edición y modificaciones en las coordenadas UV.
    \item Garantizar la persistencia de los datos recopilados entre sesiones de trabajo, integrándolos de forma segura dentro del propio archivo del proyecto.
    \item Incorporar un mecanismo explícito de consentimiento del usuario que asegure una recopilación de datos ética y transparente.
    \item Validar el correcto funcionamiento del sistema mediante una primera versión experimental con usuarios de prueba.
    \item Recopilar datos reales a partir de una versión estable del sistema utilizada por usuarios finales.
    \item Desarrollar una herramienta de análisis específica que permita explorar y analizar los datos recogidos desde una perspectiva temporal, espacial y estadística.
    \item Extraer conclusiones a partir del análisis de los datos obtenidos que permitan comprender mejor el proceso de modelado tridimensional.
\end{itemize}
