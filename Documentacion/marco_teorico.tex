\chapter{Contexto y marco teórico}

\section{Modelado tridimensional y entornos interactivos}

El modelado tridimensional es el proceso mediante el cual se crean representaciones digitales de objetos en un espacio tridimensional. Este proceso se basa en la manipulación de vértices, aristas y caras, permitiendo construir geometrías cada vez más complejas a partir de operaciones elementales.

Los entornos de modelado 3D modernos son altamente interactivos y proporcionan al usuario múltiples herramientas para modificar la geometría, aplicar transformaciones, editar superficies y gestionar escenas completas.

\section{Blender como plataforma de desarrollo}

Blender es una herramienta de código abierto ampliamente utilizada para la creación de contenido tridimensional. Además de sus capacidades de modelado, Blender ofrece un entorno extensible que permite desarrollar addons mediante Python.

\section{Registro de interacción y sistemas de logging}

El registro de interacción consiste en la recopilación sistemática de información sobre las acciones realizadas por un usuario. Permite analizar patrones de comportamiento y evolución temporal de los procesos.

\section{Análisis del proceso de creación en entornos 3D}

Analizar el proceso permite identificar fases recurrentes, dificultades comunes y estrategias utilizadas por los usuarios. Se requieren datos que reflejen la evolución temporal y estructural de los modelos.

\section{Persistencia de datos y consideraciones éticas}

Es fundamental garantizar que los datos no se pierdan entre sesiones y que se respeten principios de anonimización y consentimiento explícito del usuario.
