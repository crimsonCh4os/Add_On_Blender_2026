\chapter{Técnicas y herramientas empleadas}

\section{Entorno de desarrollo}

El sistema ha sido desarrollado utilizando \textbf{Blender} como plataforma principal de experimentación y validación. Blender es un software libre y de código abierto orientado a la creación de contenido tridimensional, que permite la extensión de sus funcionalidades mediante el desarrollo de addons en Python.

Para la implementación del sistema se ha utilizado el lenguaje \textbf{Python}, ya que constituye el lenguaje oficial de scripting de Blender y permite acceder de forma directa a su API interna (\texttt{bpy}).

\section{Arquitectura del addon}

El addon ha sido diseñado siguiendo una arquitectura modular compuesta por los siguientes componentes:

\begin{itemize}
    \item \textbf{Módulo de captura de eventos}: Encargado de interceptar y registrar eventos relevantes del proceso de modelado, tales como cambios de modo (Object/Edit), creación o eliminación de objetos, duplicados, fusiones de geometría y modificaciones UV.
    
    \item \textbf{Módulo de extracción de variables internas}: Responsable de acceder a las estructuras internas de Blender (escena, objetos, modificadores, mallas, transformaciones, etc.) y extraer variables estructurales y espaciales para su análisis posterior.
    
    \item \textbf{Módulo de análisis}: Permite procesar los datos recopilados para generar métricas temporales, espaciales y estadísticas que faciliten el diagnóstico del proceso de modelado.
    
    \item \textbf{Interfaz de usuario}: Implementada mediante paneles personalizados en la interfaz de Blender, permite gestionar el consentimiento del usuario, activar o desactivar la recopilación de datos y visualizar métricas básicas.
\end{itemize}

\section{Técnicas de extracción y análisis de datos}

Para la obtención de información estructurada se han empleado las siguientes técnicas:

\begin{itemize}
    \item \textbf{Acceso a la API interna (\texttt{bpy})}: Permite consultar propiedades de la escena, objetos, modificadores, materiales y datos de malla.
    
    \item \textbf{Uso de handlers}: Se emplean manejadores de eventos (\texttt{depsgraph\_update\_post}, \texttt{load\_post}, etc.) para detectar cambios en tiempo real dentro del entorno de modelado.
    
    \item \textbf{Serialización en formato JSON}: Empleada para estructurar los datos recogidos y facilitar su análisis posterior.
    
    \item \textbf{Análisis estadístico básico}: Cálculo de métricas como número de operaciones por sesión, frecuencia de cambios de modo, complejidad geométrica (vértices, aristas, caras), uso de modificadores y evolución temporal del modelo.
\end{itemize}

\section{Herramientas complementarias}

Además de Blender y Python, se han empleado las siguientes herramientas:

\begin{itemize}
    \item Entorno de desarrollo integrado (IDE) para la escritura y depuración del código.
    \item Sistema de control de versiones para el seguimiento del desarrollo.
    \item Librerías estándar de Python para el procesamiento de datos.
    \item Herramientas de análisis externo para el tratamiento estadístico de los datos exportados.
\end{itemize}

\section{Metodología de validación}

Para validar el correcto funcionamiento del sistema se ha seguido una metodología iterativa:

\begin{itemize}
    \item Desarrollo de una versión experimental del addon.
    \item Pruebas controladas con usuarios de prueba.
    \item Ajuste del sistema en función de los resultados obtenidos.
    \item Despliegue de una versión estable para la recopilación de datos reales.
\end{itemize}

Este conjunto de técnicas y herramientas permite la recopilación eficiente de datos del entorno de modelado y su posterior análisis estructurado para la extracción de conclusiones sobre el proceso de modelado tridimensional.
