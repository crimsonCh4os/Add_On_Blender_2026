\chapter{Introducción}

El modelado tridimensional es una actividad fundamental en numerosos ámbitos, como el desarrollo de videojuegos, la animación, la visualización arquitectónica o el diseño industrial. A pesar de la amplia adopción de herramientas de modelado 3D, el proceso mediante el cual los usuarios interactúan con estos entornos sigue siendo, en gran medida, opaco desde el punto de vista del análisis técnico y científico.

Tradicionalmente, el estudio del proceso de creación de modelos 3D se ha centrado en el resultado final, dejando en segundo plano el análisis del proceso intermedio: las decisiones tomadas por el usuario, las operaciones realizadas, el tiempo dedicado a cada fase o la evolución progresiva de la geometría. Sin embargo, comprender este proceso resulta clave para mejorar herramientas de modelado, optimizar flujos de trabajo y analizar el comportamiento de los usuarios en entornos tridimensionales. 

En este contexto, surge la necesidad de contar con sistemas capaces de registrar de forma automática y precisa la interacción del usuario dentro del entorno de modelado, sin interferir en su trabajo habitual. Dichos sistemas deben ser no intrusivos, persistentes entre sesiones y respetuosos con la privacidad del usuario, especialmente cuando se trabaja con datos reales obtenidos en contextos no controlados.

Este Trabajo de Fin de Grado presenta el diseño, implementación y validación de un sistema completo para la recopilación y análisis de datos de interacción en entornos de modelado 3D. El sistema se materializa en un addon para Blender que registra información temporal, espacial y estructural durante el proceso de modelado, almacenando los datos de forma persistente y permitiendo su posterior análisis.

El desarrollo se ha llevado a cabo de manera iterativa, comenzando con una primera versión experimental utilizada con usuarios de prueba para validar el correcto funcionamiento del sistema. Posteriormente, se ha liberado una versión estable empleada por usuarios reales, lo que ha permitido recopilar un conjunto de datos representativo para su análisis.

Como complemento al sistema de recopilación de datos, se ha desarrollado una herramienta específica para el análisis de la información obtenida, denominada \textit{Análisis 3D CSV}. Esta herramienta permite explorar los datos desde una perspectiva temporal, espacial y estadística, facilitando la extracción de conclusiones sobre el proceso de modelado 3D.

El objetivo final de este trabajo es proporcionar una base sólida para el estudio del proceso de creación de modelos tridimensionales, demostrando la viabilidad de un enfoque basado en la recopilación automática de datos reales y su análisis posterior mediante técnicas específicas adaptadas al dominio 3D.
